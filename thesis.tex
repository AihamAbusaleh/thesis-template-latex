% include alle pakete 

\documentclass[a4paper,12pt,headsepline]{scrartcl}

\usepackage{graphicx}


% Inhaltsverzechnis mit Punkten
\usepackage{tocloft}
\renewcommand{\cftsecleader}{\cftdotfill{\cftdotsep}}

% deutsche Silbentrennung
\usepackage[english]{babel}

% Umlaute unter UTF8 nutzen
\usepackage[utf8]{inputenc}

% Zeichenencoding
\usepackage[T1]{fontenc}

\usepackage{lmodern}
\usepackage{fix-cm}

%\usepackage[numbers]{natbib}
\usepackage[backend=bibtex,
				 sorting=none,
                natbib=true, 
                style=numeric,
                indexing=false,
                citereset=none,
                isbn=true,
                url=true,
                doi=true   
                ]{biblatex}
\addbibresource{mainbib.bib}
% to change format of the urldate in biblatex 
\DeclareFieldFormat{urldate}{%
  \thefield{urlday}\adddot\addspace%
  \mkbibmonth{\thefield{urlmonth}}\adddot\addspace%
  \thefield{urlyear}\isdot}
  
  
%Kopf- und Fußzeile
\usepackage{fancyhdr}
 
% mehrseitige Tabellen ermöglichen
\usepackage{longtable}

% Packet für Seitenrandabständex und Einstellung für Seitenränder
\usepackage{geometry}
\geometry{left=2.5cm, right=2cm, top=2.5cm, bottom=3cm}


% bricht lange URLs "schoen" um
%\usepackage[hyphens,obeyspaces,spaces]{url}

% Paket für Textfarben
\usepackage{xcolor}
\usepackage{hyperref}
\hypersetup{
    colorlinks=true, %set true if you want colored links
    linktoc=all,     %set to all if you want both sections and subsections linked
    linkcolor=blue,  %choose some color if you want links to stand out
       urlcolor  = blue,
            citecolor = blue,
            anchorcolor = blue
}


% Mathematische Symbole importieren
\usepackage{amssymb}

% für Tabellen
\usepackage{array}

%\bibliographystyle{plain}

% Schaltet den zusätzlichen Zwischenraum ab, den LaTeX normalerweise nach einem Satzzeichen einfügt.
\frenchspacing

% Paket für Zeilenabstand
\usepackage{setspace}

% für Listings
\usepackage{listings}
\usepackage{listingsutf8}
  % lstlisting Einstellungen 
\lstset{ 
  tabsize=2, 
    backgroundcolor=\color{lightgray}, 
  showspaces=false, 
  showstringspaces=false, 
  float=[htb], 
  captionpos=b, 
  basicstyle=\footnotesize, 
  frame=tbrl, %t: top, r, b, l 
  frameround=tttt, 
  numbers=left, 
  numberstyle=\tiny, 
  numberblanklines=false, 
   frame=single,
    breaklines=true,
 }   


% Abkürzungsverzeichnis
\usepackage[intoc]{nomencl}
\let\abk\nomenclature
\renewcommand{\nomname}{List of Abbreviations}
\renewcommand{\nomlabel}[1]{#1 \dotfill}
\setlength{\nomlabelwidth}{7cm}
\setlength{\nomitemsep}{-\parsep}
\makenomenclature

% Disable single lines at the start of a paragraph (Schusterjungen)
\clubpenalty = 10000
% Disable single lines at the end of a paragraph (Hurenkinder)
\widowpenalty = 10000
\displaywidowpenalty = 10000



% -----------------------------------------------------------------%
% -----------------------------------------------------------------%
% --------------------------DOCUMENT-------------------------------%
% -----------------------------------------------------------------%
% -----------------------------------------------------------------%


\begin{document}

 

\begin{titlepage}
\vspace*{15mm}
\begin{figure}[h]
\centering
\includegraphics[width=\textwidth]{fig/ain}
\label{fig:unilogo}
\end{figure}
{
\center
{\LARGE \textsc{Konstanz University of Applied Sciences}}\\
\vspace*{10mm}

\textbf{{\Large Bachelor Thesis}}\\
\vspace*{40mm}

\textbf{{\Large \textit{Containerization of Software Packages}}}\\ \vspace*{5mm}
{at the company }\\
\begin{figure}[h]
\centering
\includegraphics{fig/wl.jpg}
\label{fig:unilogo}
\end{figure}
\vspace*{10mm}
{\Large \textsc{Worldline Germany GmbH}}

}


\newpage
\thispagestyle{empty}

{
\center
\vspace*{15mm}
{\Huge \textbf{Bachelor Thesis}}\\ \vspace*{5mm}
\emph{for obtaining the academic degree \\ \vspace*{5mm}
{\LARGE \textbf{\textcolor{blue}{Bachelor of Science (B.Sc.) \\ Software Engineering}}} }\\
\vspace*{10mm}
{at}\\
\vspace*{10mm}
{\large \textbf{Konstanz University}}\\ \vspace*{5mm}
{Technology, Economy and Design \\ \vspace*{1mm} Faculty of Computer Science  \\ \vspace*{1mm} 
Course of studies Applied Computer Science}\\ \vspace*{10mm}
{\LARGE \textbf{\textit{Containerization of Software Packages }}} 

}
\vspace*{20mm}
\begin{tabbing}
\hspace*{4cm}\textit{Presented by:} \hspace*{1.2cm}\= Aiham Abou Saleh \\
\hspace*{4cm}\textit{Imm Nr:}  \>  292689\\
\hspace*{4cm}\textit{First supervisor:} \> Prof. Dr. Eiglsperger\\
\hspace*{4cm}\textit{Second supervisor:} \> M.Sc. Julian Weißgerber\\
\hspace*{4cm}\textit{Submission:} \> \today
\end{tabbing}
\end{titlepage}
 % 1.5 facher Zeilenabstand
\onehalfspacing


\pagestyle{plain}
\setcounter{page}{0}
\pagenumbering{roman}

% Sperrvermerk
\section*{confidentiality clause}
\textcolor{red}{
This bachelor thesis contains confidential data of company $\href{https://de.worldline.com/}{Worldline \ GmbH \ Konstanz}$.\\ \\
This work may only be made available to the first and second reviewers and authorized members of the board of examiners. Any publication and duplication of this bachelor thesis - even in part - is prohibited.\\\\
 An inspection of this work by third parties requires the expressed permission of the author and company $\href{https://de.worldline.com/}{Worldline \ GmbH \ Konstanz}$.
}
\cleardoublepage
\phantomsection

%  Abstract
\newpage
\section*{Abstract}
LATER

\newpage
% einfacher Zeilenabstand

\cleardoublepage
\phantomsection

\doublespacing
\tableofcontents
\singlespacing
\cleardoublepage
\phantomsection

% Korrigiert Nummerierung bei mehrseitigem Inhaltsverzeichnis
\newcounter{frontmatterpage}
\setcounter{frontmatterpage}{\value{page}}


% Abbildungsverzeichnis soll im Inhaltsverzeichnis auftauchen
\addcontentsline{toc}{section}{List of Figures}
\listoffigures
\cleardoublepage
\phantomsection




% Tableverzeichnis soll im Inhaltsverzeichnis auftauchen
\addcontentsline{toc}{section}{List of Tables}
\listoftables
\cleardoublepage
\phantomsection




% Listingverzeichnis soll im Inhaltsverzeichnis auftauchen
\addcontentsline{toc}{section}{Listings}
%\fancyhead[L]{Abbildungs- / Tabellen- / Listingverzeichnis} %Kopfzeile links
%\renewcommand{\lstlistlistingname}{Listingverzeichnis}
\lstlistoflistings
%\fancyhead[L]{\nouppercase{\leftmark}} %Kopfzeile links
\cleardoublepage
\phantomsection


% Abkürzungsverzeichnis

\nomenclature{RSP}{Remote Service Platforms}
\abk{IIOT}{Industrial Internet of Things }
\abk{cRSP}{common Remote Service Platform }
\abk{CCP}{Core Communication Platform}
\abk{RPM}{Red Hat Package Manager}
 \abk{CI}{Continuous Integration}
 
\printnomenclature
\cleardoublepage
\phantomsection

% 1,5 facher Zeilenabstand 
\onehalfspacing

% Nummerierung 1 , 2, 4 .. ab hier 
\setcounter{page}{0}
\pagenumbering{arabic}



\addcontentsline{toc}{section}{Introduction}
\section*{Introduction} 
\subsection*{Company}

3323
\subsection*{RSP}
2
\subsection*{IIOT}

\clearpage

% Ab hier KOpfzeile einsetzen 
\pagestyle{fancy}
\lhead{\slshape \rightmark}
\chead{}
\rhead{\slshape \leftmark}
\renewcommand{\headrulewidth}{0.5pt}


%%%%%%% einzelne Kapitels %%%%%%%%%%%%

\input{1_kap1}
\cleardoublepage
\phantomsection

\input{2_kap2}
\cleardoublepage
\phantomsection
 
 
 
%....

\input{7_summary}
\cleardoublepage
\phantomsection

\section{Conclusion}\label{conclusion}
babnabab
\cleardoublepage
\phantomsection

\newpage
%%%%%%%%%%%%%%%%%% Literaturen %%%%%%%%%%%%%%%%%%%
\addcontentsline{toc}{section}{References}
\bibliography{mainbib}
\cleardoublepage
\phantomsection

\newpage
\addcontentsline{toc}{section}{Appendix}
\subsection*{Appendices}\label{appendices}



  \lstinputlisting[language=Java]{fig/aufgabe6a.java}
  
  
\cleardoublepage
\phantomsection

\newpage
\addcontentsline{toc}{section}{Statutory Declaration}
\section*{ Statutory Declaration  }
\thispagestyle{empty}

\begin{verbatim}

\end{verbatim}

\textcolor{blue}{
I certify that this thesis does not incorporate without acknowledgement any material
previously submitted for a degree or diploma in any university; and that to the best of
my knowledge and belief it does not contain any material previously published or written
by another person where due reference is not made in the text.
}

\begin{flushleft}
\textcolor{blue}{Signature :}

\end{flushleft}

 
\begin{flushright}
\textcolor{blue}{Konstanz \today}

\end{flushright} 
 
\cleardoublepage
\phantomsection

\thispagestyle{empty} % erzeugt Seite ohne Kopf- / Fusszeile
\section*{ }

\end{document}