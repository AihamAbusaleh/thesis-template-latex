
\documentclass[a4paper,12pt,headsepline]{scrartcl}

\usepackage{graphicx}


% Inhaltsverzechnis mit Punkten
\usepackage{tocloft}
\renewcommand{\cftsecleader}{\cftdotfill{\cftdotsep}}

% deutsche Silbentrennung
\usepackage[english]{babel}

% Umlaute unter UTF8 nutzen
\usepackage[utf8]{inputenc}

% Zeichenencoding
\usepackage[T1]{fontenc}

\usepackage{lmodern}
\usepackage{fix-cm}

%\usepackage[numbers]{natbib}
\usepackage[backend=bibtex,
				 sorting=none,
                natbib=true, 
                style=numeric,
                indexing=false,
                citereset=none,
                isbn=true,
                url=true,
                doi=true   
                ]{biblatex}
\addbibresource{mainbib.bib}
% to change format of the urldate in biblatex 
\DeclareFieldFormat{urldate}{%
  \thefield{urlday}\adddot\addspace%
  \mkbibmonth{\thefield{urlmonth}}\adddot\addspace%
  \thefield{urlyear}\isdot}
  
  
%Kopf- und Fußzeile
\usepackage{fancyhdr}
 
% mehrseitige Tabellen ermöglichen
\usepackage{longtable}

% Packet für Seitenrandabständex und Einstellung für Seitenränder
\usepackage{geometry}
\geometry{left=2.5cm, right=2cm, top=2.5cm, bottom=3cm}


% bricht lange URLs "schoen" um
%\usepackage[hyphens,obeyspaces,spaces]{url}

% Paket für Textfarben
\usepackage{xcolor}
\usepackage{hyperref}
\hypersetup{
    colorlinks=true, %set true if you want colored links
    linktoc=all,     %set to all if you want both sections and subsections linked
    linkcolor=blue,  %choose some color if you want links to stand out
       urlcolor  = blue,
            citecolor = blue,
            anchorcolor = blue
}


% Mathematische Symbole importieren
\usepackage{amssymb}

% für Tabellen
\usepackage{array}

%\bibliographystyle{plain}

% Schaltet den zusätzlichen Zwischenraum ab, den LaTeX normalerweise nach einem Satzzeichen einfügt.
\frenchspacing

% Paket für Zeilenabstand
\usepackage{setspace}

% für Listings
\usepackage{listings}
\usepackage{listingsutf8}
  % lstlisting Einstellungen 
\lstset{ 
  tabsize=2, 
    backgroundcolor=\color{lightgray}, 
  showspaces=false, 
  showstringspaces=false, 
  float=[htb], 
  captionpos=b, 
  basicstyle=\footnotesize, 
  frame=tbrl, %t: top, r, b, l 
  frameround=tttt, 
  numbers=left, 
  numberstyle=\tiny, 
  numberblanklines=false, 
   frame=single,
    breaklines=true,
 }   


% Abkürzungsverzeichnis
\usepackage[intoc]{nomencl}
\let\abk\nomenclature
\renewcommand{\nomname}{List of Abbreviations}
\renewcommand{\nomlabel}[1]{#1 \dotfill}
\setlength{\nomlabelwidth}{7cm}
\setlength{\nomitemsep}{-\parsep}
\makenomenclature

% Disable single lines at the start of a paragraph (Schusterjungen)
\clubpenalty = 10000
% Disable single lines at the end of a paragraph (Hurenkinder)
\widowpenalty = 10000
\displaywidowpenalty = 10000

